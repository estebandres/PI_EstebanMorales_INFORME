% Chapter Template

\chapter{Conclusiones y Trabajos Futuros} % Main chapter title

\label{Chapter13} % Change X to a consecutive number; for referencing this chapter elsewhere, use \ref{ChapterX}

\steveCabecera{Capítulo 13. \emph{Conclusiones y Trabajos Futuros}} % Change X to a consecutive number; this is for the header on each page - perhaps a shortened title

%----------------------------------------------------------------------------------------
%	SECTION 1
%----------------------------------------------------------------------------------------
\section{Conclusión}
Al finalizar el proyecto se obtuvo una aplicación android que implementa los casos de uso documentados y cumple con los requerimientos del producto. 
La realización de este proyecto en particular implicó un amplio conocimiento sobre los diversos protocolos y estándares de comunicación involucrados en su funcionamiento.
 
Durante el desarrollo se hizo evidente la pronunciada curva de aprendizaje asociada con la implementación de Clean Architecture sobre una aplicación android en Java. Encima de esta complejidad la incorporación del paradigma reactivo utilizando RxJava lo hizo aún más cuesta arriba.
Afortunadamente todas las dificultades fueron sorteadas y el desarrollo se completó con éxito.

En cuanto al empleo del patrón de arquitectura se identificaron algunas desventajas y ventajas.
Las reglas establecidas introducen una rigidez en la codificación que parece contradecir el principio KISS de la programación. Esta situación puede llevar a al empleo de workarounds que no siempre son óptimos y que podrían resultar en errores. Al mismo tiempo la cantidad de código de andamiaje hace que la navegación por los archivos del proyecto no sea obvia.\\
Sin embargo, una vez familiarizado con la estructura del proyecto, las adición de funcionalidades y las modificaciones en general se convierten en un proceso sistemático y repetitivo al estilo de una receta de cocina. La división de responsabilidades definidas facilita enormemente la inspección del código y las tareas de debugging.

Definitivamente, el uso de este patrón de arquitectura se justifica cuando es necesario coordinar el trabajo con muchos integrantes en proyectos de una mayor envergadura.

Como parte del desarrollo se escribieron pruebas unitarias y de interfaz gráfica. Esto requirió familiaridad con conceptos como los dobles de prueba y su codificación utilizando los frameworks disponibles. Al momento de concluir con el proyecto la cobertura de las pruebas unitarias alcanzaba los casos de uso más importantes y las clases de la capa de datos.

En una nota personal, la realización de este proyecto integrador fue una experiencia sumamente enriquecedora, que me permitió adquirir conocimientos en una amplio abanico de conceptos. Más allá de la profunda formación autodidacta que resultó en un producto de software con estándares industriales, la lección más valiosa fue reconocer el potencial adquirido a lo largo de la carrera de poder proponer, diseñar e implementar soluciones para problemas del mundo real.

\section{Trabajos Futuros}
Se sugiere utilizar un nuevo canal de comunicación a través de Bluetooth para realizar la configuración inicial de un nuevo módulo.

Por razones de tiempo no se pudo explorar las opciones de securitización sobre el canal de comunicación por internet. 
queda pendiente la configuración de políticas de acceso utilizando la API HTTP provista por EMQX.

También podría evaluarse el aprovisionamiento de certificados únicos para instalaciones de la aplicación así como para el stock de módulos electrónico
de manera que se garantice que solo los dispositivos autorizados pueden utilizar el servicio del broker MQTT.

En cuanto a funcionalidades para la aplicación algunos de los usuarios sugirieron la adición de grupos de módulos que se puedan accionar en simultáneo.
A simple vista al emplear MQTT se podría resolver de una manera sencilla agregando nuevos tópicos por grupos de módulos por usuario.



