% Appendix A

\chapter{Código} % Main appendix title

\label{AppendixA} % For referencing this appendix elsewhere, use \ref{AppendixA}

\lhead{Apéndice A1. \emph{Alarma}} % This is for the header on each page - perhaps a shortened title
\section{{\emph{Alarma}}}
Para ver el código implementado en la alarma, ingresar en el siguiente link:

\href{url}{https://bitbucket.org/santiagosalamandri/alarma\_1.5}

\subsection{Referencias utilizadas}
\begin{itemize}
\item Código:
	\begin{itemize}
	\item MQX RTOS\cite{embedded1}.
	\end{itemize} 
\item Diseño: 
	\begin{itemize}
	\item Diseño de alarma hogareña \cite{alarm1}.
	\item Como diseñar alarma \cite{alarm2}.
	\item Alarma hogareña Inalambrica\cite{alarm3}.
	\end{itemize}
	
\item Comunicación:
	\begin{itemize}
	\item TCP/IP Protocol Design \cite{socket1}.
	\item Socket\cite{socket9}.
	\end{itemize}
\item Circuito impreso:
	\begin{itemize}
	\item Kicad design\cite{pcb1}.
	\end{itemize}
\end{itemize}

\lhead{Apéndice A2. \emph{Servidor}} % This is for the header on each page - perhaps a shortened title
\section{{\emph{Servidor}}}
Para ver el código implementado en la servidor, ingresar en el siguiente link:

\href{url}{https://bitbucket.org/santiagosalamandri/web\_server3}
\subsection{Referencias utilizadas}
\begin{itemize}
\item Código:
\begin{itemize}
\item Desarrollo ágil Web \cite{rails1Book}.
\item Rails \cite{rails2Book}.
\item Ruby\cite{ruby1Book}.
\item Rails y Bootstrap\cite{rails1}.
\item JavaScript en Rails\cite{rails2}.
\item Módulo \textit{e-mail} \cite{rails3}.
\end{itemize}
		
\item Comunicación:
\begin{itemize}
\item BSD Sockets en Ruby\cite{socket2}.
\item Programación de Sockets en Ruby  \cite{socket3}.
\item Ruby Sockets \cite{socket4}.
\item Programación de Sockets\cite{socket5}.
\item Documentación de sockets en Ruby\cite{socket10}.
\end{itemize}
\item Base de datos:
	\begin{itemize}
	\item Queries SQL en Ruby\cite{sql1}
	\end{itemize}
\item \textit{Front-end}:
	\begin{itemize}
	\item Libro de Ajax\cite{ajax1Book}.
	\item Ajax en Ruby \cite{ajax1}.
	\item Tutorial de Ajax en Ruby \cite{ajax2}.
	\item Ajax y Boostrap \cite{ajax3}.
	\end{itemize}
\end{itemize}


\lhead{Apéndice A3. \emph{Aplicación}} % This is for the header on each page - perhaps a shortened title
\section{{\emph{Aplicación}}}

Para ver el código implementado en la aplicación, ingresar en el siguiente link:

\href{url}{https://bitbucket.org/santiagosalamandri/app-alarma-final}

\subsection{Referencias utilizadas}
\begin{itemize}
\item Código:
\begin{itemize}
\item Libro Desarrollo en Android\cite{android1Book}.
\item Tutorial Android \cite{android1}.
\item Tutorial Android Studio \cite{android2}.
\item Curso Programación Android \cite{android3}.
\end{itemize}

\item Comunicación:
\begin{itemize}
\item Ejemplos de Android Socket\cite{socket6}.
\item Tutorial de Android TCP conections \cite{socket7}.
\item Sockets en Android \cite{socket8}. 
\end{itemize}

\end{itemize}

\section{{\emph{PCB}}}
Para ver el proyecto del \textit{PCB} de la alarma, ingresar en el siguiente link:

\href{url}{https://bitbucket.org/santiagosalamandri/relay\_shield}
